\documentclass[11pt]{article}

\title{\textbf{Fallen Kingdom}}
\author{by Falcobuster}
\date{}

\usepackage[margin=0.5in]{geometry}
\usepackage{ocgx}
\usepackage{textcomp}
\usepackage{underscore}
\usepackage{graphicx}
\usepackage[utf8]{inputenc}

\begin{document}

\maketitle

Version 1.6

\section{Emulator Setup (ParallelN64)}

The recommended emulator for this hack (and N64 ROMS in general) is ParallelN64, a core for RetroArch. ParallelN64 is faster, more accurate, less buggy, more configurable, and supports better graphics plugins than Project64; however, the RetroArch interface is notoriously confusing and annoying to use. For this reason, I am creating a launcher for ParallelN64 that allows to you bypass RetroArch's interface entirely. Unfortunately, if you just downloaded this hack and are reading this, it hasn't been released yet, so you will need to manually set up RetroArch using the instructions below.

Alternatively, you \textit{can} use Project64 if you use a specific graphics plugin, but you will likely experience minor lag in some parts, and there may be some minor graphical bugs due to the Project64 version of the plugin being 8 years older than the ParallelN64 version.
\bigbreak
Download and install RetroArch, then go to \textbf{Main Menu \textrightarrow Load Core \textrightarrow Download a Core} and scroll down to find \textit{Nintendo - Nintendo 64 (ParaLLEl N64)} and install it.
\bigbreak
First, go to \textbf{Settings \textrightarrow Configuration} and make sure that \textbf{Save Configuration on Quit} is set to \textbf{ON}
\bigbreak
Next, go to \textbf{Settings \textrightarrow Video} and set the following settings:
\begin{itemize}
	\item \textbf{Suspend Screensaver} set to \textbf{ON}
	\item \textbf{Windowed Mode}
	\begin{itemize}
		\item \textbf{Remember Window Position and Size} set to \textbf{ON}
		\item \textbf{Window Width} and \textbf{Window Height} set to integer multiples of 320x240 (eg. 1280x960)
	\end{itemize}
	\item \textbf{Scaling}
	\begin{itemize}
		\item \textbf{Integer Scale} set to \textbf{ON}
	\end{itemize}
	\item \textbf{Synchronization}
	\begin{itemize}
		\item \textbf{Vertical Sync (Vsync)} set to \textbf{OFF}
	\end{itemize}
\end{itemize}
\bigbreak
Now to setup your controller, go to \textbf{Settings \textrightarrow Input \textrightarrow Port 1 Controls}. RetroArch uses a confusing setup where controllers are bound to a virtual controller, which is then bound again for each core. To make things simpler, you can just map the following inputs so they should automatically work with N64:
\begin{itemize}
	\item \textbf{B Button (Down)} is the N64 \textbf{A Button}
	\item \textbf{Y Button (Left)} is the N64 \textbf{B Button}
	\item \textbf{Start Button} is actually the \textbf{Start Button}
	\item \textbf{D-Pad} maps as expected to the N64 \textbf{D-Pad}
	\item \textbf{L Button (Shoulder)} is the N64 \textbf{L Trigger}
	\item \textbf{R Button (Shoulder)} is the N64 \textbf{R Trigger}
	\item \textbf{L2 Button (Trigger)} is the N64 \textbf{Z Trigger}
	\item \textbf{Left Analog} maps to the N64 \textbf{Analog Stick} as expected
	\item \textbf{Right Analog} maps to the N64 \textbf{C Buttons} as expected
\end{itemize}
\bigbreak
Finally, launch an N64 ROM, then press \textbf{F1} after starting the ROM to bring up the \textbf{Quick Menu}. Now go to \textbf{Options} and set the following settings:
\begin{itemize}
	\item \textbf{CPU Core} set to \textbf{dynamic_recompiler}
	\item \textbf{Enable Expansion Pak RAM} set to \textbf{ON}
	\item \textbf{GFX Accuracy} set to \textbf{veryhigh}
	\item \textbf{(ParaLLEl-RDP) Upscaling Factor} set to \textbf{4x} (or whatever integer scale you used earlier for your window size)
	\item \textbf{(ParaLLEl-RDP) Use native texture LOD when upscaling} set to \textbf{ON}
	\item \textbf{GFX Plugin} set to \textbf{glide64} (NOTE: for console compatible hacks- which this is not- \textbf{parallel} is the best choice instead)
	\item \textbf{Resolution} set to the same as you set for the window size
	\item \textbf{VI Refresh (Overclock)} set to \textbf{2200}
	\item \textbf{Framerate} set to \textbf{fullspeed}
\end{itemize}
\bigbreak
Now, close the emulator and restart RetroArch. Your settings are now good for all N64 roms.
	
\section{Emulator Setup (Project64)}
Project64 is semi-obsolete and not recommended, but it is still usable for this hack. If you are using Project64, then you must use graphics plugin included in the \textbf{For PJ64 Users} folder. Go to your Project64 directory (which is probably C:\textbackslash{}Program Files (x86)\textbackslash{}Project64 2.4) and place the \textbf{glide3x.dll} file there. Then, copy the three files in the included \textbf{Plugins} subdirectory into the \textbf{Plugins} directory in your Project 64 folder. If you did everything correctly, the plugin should appear in your list of graphics plugins in Project64 as \textit{Glide64 Final Date May 8 2012}.
	Now, open up Project 64 and go to \textbf{Options \textrightarrow Settings} and uncheck the box that says \textbf{Hide advanced settings}. Now set the following settings under \textbf{Defaults}:
\begin{itemize}
	\item \textbf{Memory size} set to \textbf{8 MB}
	\item \textbf{Counter factor} set to \textbf{1}
	\item \textbf{VI refresh rate} set to \textbf{2200}
	\item \textbf{Unaligned DMA} checked
\end{itemize}
Next, go to \textbf{Plugins} and select \textbf{Glide64 Final Date May 8 2012} from the \textbf{Video (graphics) plugin} combo box.

\section{Graphics Plugin Test}
For your convenience, I have included a graphics plugin test on the first screen of the ROMhack. If your plugin is correct, it should look like this (but the static will be much lower quality on Project64):
\bigbreak
\includegraphics[width=\linewidth]{{"./blend/levels/Emulator Check/correct-display"}.png}
\bigbreak
If the emulator check screen shows CPU Clock Emulation Check: {\color{red}FAIL}, this means you have not set \textbf{Framerate} to \textbf{fullspeed} (or in the case of Project 64, set \textbf{Counter Factor} to \textbf{1}). It is important to get this setting right, otherwise you will likely have performance issues. If you are using Project 64, you may need to completely close the ROM after changing the settings, then open it again.

\section{Gameplay}

This hack has some challenging platforming, but nothing too egregious. If you are trying to do something that involves a first or second frame wallkick or an absurdly precise angle, then what you are attempting to do is either impossible or an unintended sequence break. Some places may require you to get an item first before you can navigate them.

There are 10 golden mushrooms hidden throughout the game, as well as 3 life shrooms. Each golden shroom increases your maximum health by half a heart, and each life shroom increases your maximum health by a full heart. If you collect all 10 golden shrooms and all 3 life shrooms, you will have 12 hearts by the end of the game. If you want to know in which levels the shrooms are located, see \textbf{hints.htm}.

If you are having trouble with a certain part of the game, you can look at \textbf{hints.htm} for tips.

\section{Music Credits}

\fontsize{10}{12}\selectfont

\textit{\Large Twilight Realm}\\
\textbf{\small Source Game:} Twilight Princess\\
\textbf{\small Use in ROM:} File Select Theme\\
\textbf{\small Composer:} Asuka Ohta, Koji Kondo, and Toru Minegishi\\
\textbf{\small M64 Sequencer:} ShrooboidBrat\\
\break
\textit{\Large Lost Emotion}\\
\textbf{\small Source Game:} Touhou 14.5: Urban Legend in Limbo\\
\textbf{\small Use in ROM:} Chasm of Lost Hope Theme\\
\textbf{\small Composer:} Jun'ya ``ZUN'' Ōta\\
\textbf{\small M64 Sequencer:} pieordie1\\
\break
\textit{\Large Faraway Voyage of 380,000 Kilometers}\\
\textbf{\small Source Game:} Touhou 15: Legacy of Lunatic Kingdom\\
\textbf{\small Use in ROM:} Star Road Theme\\
\textbf{\small Composer:} Jun'ya ``ZUN'' Ōta\\
\textbf{\small M64 Sequencer:} pieordie1\\
\break
\textit{\Large Dark Overworld}\\
\textbf{\small Source Game:} A Link to the Past\\
\textbf{\small Use in ROM:} Great Plains Theme\\
\textbf{\small Composer:} Koji Kondo\\
\textbf{\small M64 Sequencer:} cpuHacka101\\
\break
\textit{\Large Riding the Shell}\\
\textbf{\small Source Game:} Rayman 2\\
\textbf{\small Use in ROM:} Restless Rapids Theme\\
\textbf{\small Composer:} 	Eric Chevalier\\
\textbf{\small M64 Sequencer:} pieordie1\\
\break
\textit{\Large Waterfall}\\
\textbf{\small Source Game:} Undertale\\
\textbf{\small Use in ROM:} Flooded Temple Theme\\
\textbf{\small Composer:} Toby Fox\\
\textbf{\small M64 Sequencer:} MariosHub\\
\break
\textit{\Large Lower Norfair}\\
\textbf{\small Source Game:} Super Metroid\\
\textbf{\small Use in ROM:} Tal Tal Mines Theme\\
\textbf{\small Composer:} Kenji Yamamoto and Minako Hamano\\
\textbf{\small M64 Sequencer:} cpuHacka101\\
\break
\textit{\Large Boss Theme}\\
\textbf{\small Source Game:} Majora's Mask\\
\textbf{\small Use in ROM:} Blaarg Ambush Miniboss Theme\\
\textbf{\small Composer:} Koji Kondo\\
\textbf{\small M64 Sequencer:} EDark\\
\break
\textit{\Large Solar Sect of Mystic Wisdom $\sim$ Nuclear Fusion}\\
\textbf{\small Source Game:} Touhou 11: Subterranean Animism\\
\textbf{\small Use in ROM:} Moldorm Boss Theme\\
\textbf{\small Composer:} Jun'ya ``ZUN'' Ōta\\
\textbf{\small M64 Sequencer:} pieordie1\\
\break
\textit{\Large Reach for the Moon, Immortal Smoke}\\
\textbf{\small Source Game:} Touhou 8: Imperishable Night\\
\textbf{\small Use in ROM:} Bamboo Forest of the Lost Theme\\
\textbf{\small Composer:} Jun'ya ``ZUN'' Ōta\\
\textbf{\small M64 Sequencer:} pieordie1\\
\break
\textit{\Large Secret Course}\\
\textbf{\small Source Game:} Super Mario Sunshine\\
\textbf{\small Use in ROM:} Bamboo Forest of the Lost Secret World Theme\\
\textbf{\small Composer:} 	Koji Kondo and Shinobu Tanaka\\
\textbf{\small M64 Sequencer:} DobieMeltfire\\
\break
\textit{\Large Pure Furies $\sim$ Whereabouts of the Heart}\\
\textbf{\small Source Game:} Touhou 15: Legacy of Lunatic Kingdom\\
\textbf{\small Use in ROM:} Bowser's Theme\\
\textbf{\small Composer:} Jun'ya ``ZUN'' Ōta\\
\textbf{\small M64 Sequencer:} pieordie1\\



\end{document}
